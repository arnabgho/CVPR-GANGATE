\section{Introduction}
Generative methods have made huge strides in the last few years driven by the success of Variational Autoencoders (VAEs)~\cite{kingma2013auto} and Generative Adversarial Networks (GANs)~\cite{goodfellow2014generative}. 
While these methods can generate impressive results, one major challenge is that quality degrades quickly when the diversity of training data (e.g., number of object/scene classes) increases, especially when the manifold of the multi-class distribution of images drawn from the ground truth distribution is not continuous. \ow{um... I don't really get this previous sentence}

We propose a solution to generate multi-class images using a form of class conditioning that outperforms prior class conditioned GANs while requiring significantly fewer parameters than using a single GAN per-class. 
To do this, we take advantage of the fact that different classes will share similar visual features, and propose an architecture that learns to pick a subset of parameters to use for each class.
Our method leverages a fully-residual GAN architecture, both in the generator and discriminator.
This is combined with a second, ``gating'' network to determine which set of generator and discriminator blocks to use conditioned on the input class. \ow{explanation gets harder if ``our'' method is channel-wise}

This gating network serves to perform a form of class conditioning, and we compare the effect of it to a number of other forms of conditioning, such as for example concatenating one-hot class information, auxillary classification, \todo{...}.
\ow{describe high level idea and outcome of the 1d experiments}

We evaluate our gating approach on a number of applications \todo{...}
\ow{discuss evaluation and findings}

In summary, our Contributions are :
\begin{itemize}
\item Incision experiments on a trained Residual Generator based GAN on the 1D Mixture of Gaussians to show certain blocks correspond to certain modes in the generated distribution.
\item Introduction of the Gated Residual Blocks and the Hypernetwork to predict the corresponding alphas on the Generator and Discriminator.
\item Introduction of a new task based on Generative Adversarial Networks namely the task of generating high resolution realistic images from very rough and sparse outline like scribbles.
\end{itemize}